\documentclass[a4paper]{ctexrep}
\usepackage{listings, graphicx, color, soul, xcolor, bm, geometry, fancyhdr, makeidx,CJK, fontspec, amsmath, savesym}
\usepackage[colorinlistoftodos]{todonotes}
\usepackage[colorlinks,linkcolor=black]{hyperref}
\setmonofont[Mapping={}]{Monaco} %英文引号之类的正常显示,相当于设置英文字体
\setsansfont{Monaco} %设置英文字体 Monaco, Consolas,  Fantasque Sans Mono
\setmainfont{Monaco} %设置英文字体
% \setCJKmainfont{方正兰亭黑简体}  %中文字体设置
% \setCJKsansfont{华康少女字体} %设置中文字体
% \setCJKmonofont{华康少女字体} %设置中文字体

% \savesymbol{iint}
% \usepackage{txfonts} % txfonts包和amsmath包中的\iint符号冲突,需要先用savesym包保存符号后还原
% \restoresymbol{TXF}{iint}


\definecolor{mygreen}{rgb}{0,0.6,0}
\definecolor{mygray}{rgb}{0.5,0.5,0.5}
\definecolor{mymauve}{rgb}{0.58,0,0.82}
\lstset{ %
backgroundcolor=\color{white},   % choose the background color
basicstyle=\footnotesize\ttfamily,        % size of fonts used for the code
columns=fullflexible,
breaklines=true,                 % automatic line breaking only at whitespace
captionpos=b,                    % sets the caption-position to bottom
tabsize=4,
commentstyle=\color{mygreen},    % comment style
escapeinside={\%*}{*)},          % if you want to add LaTeX within your code
keywordstyle=\color{blue},       % keyword style
stringstyle=\color{mymauve}\ttfamily,     % string literal style
frame=single,
rulesepcolor=\color{red!20!green!20!blue!20},
% identifierstyle=\color{red},
language=c++,
}

\newcommand{\HRule}{\rule{\linewidth}{0.5mm}} % Defines a new command for the horizontal lines,
\geometry{left=3cm,right=3cm,top=3cm,bottom=3cm}
\makeindex
\pagestyle{headings}
\pagestyle{fancy}
\fancyhf{}
\fancyhead[C]{\textbf{By Netcan}}
\fancyhead[R]{\leftmark}
\fancyfoot[ER,OR]{\normalsize\thepage}

\begin{document}

\begin{titlepage}
\newgeometry{top=6.4cm}
\begin{center}
\includegraphics{acm_icpc_logo.png}\\[0.5cm]
\HRule\\[0.4cm]
{\huge \bfseries ACM-ICPC Template Libraries}\\[0.2cm]
\HRule\\[1.5cm]
\textsc{\large 合肥工业大学宣城校区}\\[1cm]
\includegraphics[width=3cm, height=3cm]{hfut_logo.jpg}\\[0.5cm]
Author: \textbf{\emph{Netcan}}\\
Blog: \textbf{\emph{\url{http://www.netcan.xyz}}}\\
{\large \today}\\[2cm]
\end{center}
\end{titlepage}

\tableofcontents

\chapter{数学}
\section{$0-20$的阶乘}
\begin{lstlisting}
const long long fac[21]={1,1,2,6,24,120,720,5040,40320,362880,
						3628800,39916800,479001600,6227020800,
						87178291200,1307674368000,20922789888000,
						355687428096000,6402373705728000,121645100408832000,
						2432902008176640000};
\end{lstlisting}
\section{错排公式}
有n个元素的排列,若一个排列中所有的元素都不在自己原来的位置上,错排数记为$D(n)$,则
$$D(n) = (n-1)[D(n-1)+D(n-2)]$$
\section{最小公倍数$lcm(a,b)$ \&\& 最大公约数$gcd(a,b)$}
\begin{lstlisting}
inline int gcd(int a, int b) {
	return b==0?a:gcd(b,a%b)
}
inline int lcm(int a, int b) {
	return a/gcd(a,b)*b;
}
\end{lstlisting}
\section{母函数}
$G(x)=(1+x+x^2+\dots+x^N)(1+x^2+x^4+\dots+x^N)\dots(1+x^N)$展开后$x^N$的系数\underline{(注意溢出)}\\
%\begin{lstlisting}
%\end{lstlisting}
\begin{lstlisting}
int c1[MAX_N], c2[MAX_N]; // c1表示每一项的的系数,c2表示每个表达式的临时系数
for(int i=0; i<=N; ++i) { // 每一项应该初始化为1,即1+x+x^2+...+x^N
	c1[i] = 1;
	c2[i] = 0;
}
for(int i=2; i<=N; ++i) { // 从第二个表达式开始
	for(int j=0; j<=N; ++j) // 表示第一个表达式的第j项
		for(int k=0;k+j<=N;k+=i) // k表示后一个表达式的第k项
			c2[j+k] += c1[j]; // 这里应该是相当于C1*x^j * x^k=C1*x^(j+k),即c2[k+j]+=C1[j]
	for(int j=0; j<=N; ++j) {
		c1[j] = c2[j]; // 确定x^j的系数
		c2[j] = 0;
	}
}
\end{lstlisting}
\section{动态规划}
\subsection{最长公共子序列}
$$
dp[i+1][j+1]=\begin{cases}
	dp[i][j]+1 & s1[i]=s2[j] \cr
	max(dp[i+1][j], dp[i][j+1]) & s1[i] \neq s2[j]
\end{cases}
$$
$$
op[i+1][j+1]=\begin{cases}
	\nwarrow & s1[i]=s2[j] \cr
	\uparrow & dp[i][j+1] \geq dp[i+1][j] \cr
	\leftarrow & dp[i][j+1] < dp[i+1][j] \cr
\end{cases}
$$
\subsection{最长上升子序列}
$$dp[i]=max\{1, dp[j]+1 | j<i \&\& a_j<a_i\}$$
$O(N^2)$算法

\begin{lstlisting}
int a[MAX_N], dp[MAX_N]; // a为序列, dp[i]为以a[i]为结尾的最长上升子序列长度
int Maxlis = 0; // 最长上升子序列长度
int n; // a的有效长度
for(int i=0; i<n; ++i) {
	dp[i] = 1; // 记得初始化为1
	for(int j=0; j<i; ++j)
		if(a[i] > a[j])
			dp[i] = max(dp[i], dp[j] + 1);
	Maxlis = max(Maxlis, dp[i]); // 记得更新Maxlis
}
\end{lstlisting}
$O(n log(n))$算法

\begin{lstlisting}
#define INF 0x3f3f3f3f
int a[MAX_N], dp[MAX_N]; // a为序列, dp[i]存放长度为i+1的上升序列中末尾元素的最小值
int n; // 序列长度
memset(dp, 0x3f, sizeof(dp)); // 初始化dp[i]值都为INF
for(int i=0; i<n; ++i)
	*lower_bound(dp, dp+n, a[i]) = a[i];
// cout << lower_bound(dp, dp+n, INF) - dp << endl; // 最长上升序列的长度
\end{lstlisting}

$lower\_bound(dp, dp+n, k)$这个函数从已经排序好的序列$dp$中,二分搜索找出满足$dp_i \ge k$的$dp_i$的最小指针。

$upper\_bound(dp, dp+n, k)$这个函数则从已经排序好的序列$dp$中,二分搜索找出满足$dp_i > k$的$dp_i$的最小指针。

例如求有序数组$a$中的$k$的个数,可以利用下面的代码求出:
\begin{lstlisting}
upper_bound(a, a+n, k) - lower_bound(a, a+n, k)
\end{lstlisting}





\section{高精度}
\subsection{加法\&\&乘法}
适合大数的加法和乘法
\begin{lstlisting}
struct BigInt {
	const static int nlen = 4; // 控制每个数组数字长度,默认为4,计算乘法的时候每个数组相乘也不会溢出int范围
	const static int mod = 10000; // 值为10^nlen
	short n[1000], len;  // 最多存4*1000位长度,可调,short占的内存小,但是速度慢
	BigInt() {
		memset(n, 0, sizeof(n));
		len = 1;
	}
	BigInt(int num) {
		len = 0;
		while(num >0) {
		n[len++] = num%mod;
		num/=mod;
		}
	}
	BigInt(const char *s) {
		int l = strlen(s);
		len = l % nlen  == 0 ? l/nlen : l/nlen+1;
		int index = 0;
		for(int i=l-1; i>=0; i -= nlen) {
		int tmp = 0;
		int j = i-nlen+1;
		if(j<0) j = 0;
		for(int k=j; k<=i; ++k)
			tmp = tmp*10+s[k]-'0';
		n[index++] = tmp;
		}
	}
	BigInt operator+(const BigInt &b) const { // 加法
		BigInt res;
		res.len = max(len, b.len);
		for(int i=0; i<res.len; ++i) {
		res.n[i] += (i<len ? n[i]:0) + (i<b.len ? b.n[i]:0);
		res.n[i+1] += res.n[i]/mod;
		res.n[i] = res.n[i]%mod;
		}
		if(res.n[res.len] > 0) ++res.len;
		return res;
	}
	BigInt operator*(const BigInt &b) const { // 乘法
		BigInt res;
		for(int i=0; i<len; ++i) { // 类似母函数,第一个数组
		int up = 0; // 进位
		for(int j=0; j<b.len; ++j) { // 第二个数组
			int tmp = n[i]*b.n[j] + up + res.n[i+j]; // 控制nlen=4是防止tmp溢出
			res.n[i+j] = tmp%mod;
			up = tmp/mod;
		}
		if(up!=0)
			res.n[i+b.len] = up;
		}
		res.len = len+b.len;
		while(res.n[res.len-1] == 0 && res.len>1 ) --res.len;
		return res;
	}
	void show() const {
		printf("%d", n[len-1]); // 先输出最高位,后面可能需要前导0
		for(int i=len-2; i>=0; --i)
		printf("%04d", n[i]); // 前导0,%04d和nlen一致
		printf("\n");
	}
};
\end{lstlisting}



\chapter{计算几何}
\section{点}


\chapter{组合博弈}
\section{SG函数 \&\& NIM游戏}
\begin{enumerate}
	\item 可选步数为$[1, m]$的连续整数,直接取模即可,$SG(x)=x\%(m+1)$;
	\item 可选步数为任意步,$SG(x)=x$;
\end{enumerate}

步数集合S需排序(升序), sg数组记得初始化,对于确定的步数,一系列sg(x)值也就确定了。

最终判断各个sg(x)的异或和,即可判断胜负。异或和为0先手必败,反之必胜。

\subsection{打表}
\begin{lstlisting}
int S[STEP_N], steps, sg[MAX_N];  // S集合存放走法,steps存放走法数,sg存放sg(x)的值
bool vis[MAX_N]; // 标记
void get_sg(int n) {
	memset(sg, 0, sizeof(sg)); // 初始化sg
	for(int i=1; i<=n; ++i) { // 从sg[1]开始计算
		memset(vis, 0, sizeof(vis)); // 每次计算完一个sg值需要归零
		for(int j=0; S[j] <= i && j <steps ; ++j)
			vis[sg[i-S[j]]] = true; // 标记各个后继节点的sg值
		for(int j=0; j<=n; ++j)
			if(!vis[j]) { // 找出sg补集的最小值
				sg[i] = j;
				break;
			}
	}
}
\end{lstlisting}
\subsection{递归}
\begin{lstlisting}
#pragma comment(linker, "/STACK:1024000000,1024000000") // 防止爆栈
int S[STEP_N], sg[MAX_N], k; // 题目中的步数集合S, 以及sg(t)函数值sg(t), 步数集合大小k

int SG(int p) { // 求sg(t)值函数SG(t)
	bool vis[101] = {false}; // 标记各个sg(t)的值,为了方便求补集最小值(sg(t)),数组不宜开过大,爆栈就扩栈
	for(int i=0; i<k; ++i) {
		int t = p - S[i];
		if(t < 0) // 小于0则退出循环,求出该层的sg(t)值
			break;
		if(sg[t] == -1) // 记得memset(sg, -1, sizeof(sg));
			sg[t] = SG(t); // 递归求sg(t)
		vis[sg[t]] = true; // 标记该层的sg(t)值
	}
	for(int i=0;; ++i) // 求出该层的sg(t)值,即补集的最小值
		if(!vis[i])
			return i;
}
\end{lstlisting}



\chapter{数据结构}
\section{二叉搜索树}
二叉搜索树是能够高效地进行如下操作的数据结构:
\begin{itemize}
		\item 插入一个数值
		\item 查询是否包含某个数值
		\item 删除某个数值
\end{itemize}

时间复杂度:\(O(log(n))\)
\begin{lstlisting}
struct node { // 树节点
		int val;
		node *lch, *rch;
};

node *insert(node *p, int x) { //插入数值x
	if(p == NULL) { // 新建节点插入
		node *q = new node;
		q->val = x;
		q->lch = q->rch = NULL;
		return q;
	}
	else {
		if(x < p->val)	p->lch = insert(p->lch, x); // 往左边搜索
		else	p->rch = insert(p->rch, x); // 往右边搜索
		return p;
	}
}

bool find(node *p, int x) { // 查找数值x
	if(p == NULL) return false; // 找不到
	else if(p->val == x) return true; // 找到
	else if(x < p->val) return find(p->lch, x); // 往左边搜索
	else return find(p->rch, x); // 往右边搜索
}

node *remove(node *p, int x) { // 删除数值x
	if(p == NULL) return NULL; // 找不到数值
	else if(x < p->val) p->lch=remove(p->lch, x); // 往左边搜索
	else if(x > p->val) p->rch=remove(p->rch, x); // 往右边搜索
	else { // 找到
		if(p->lch == NULL) { // 如果删除的节点没有左儿子,将右儿子提上来
			node *q = p->rch;
			delete p; // 删除
			return q;
		}
		else if(p->lch->rch == NULL) { // 如果删除的节点左儿子没有右儿子,将左儿子提上来
			node *q = p->lch;
			q->rch = p->rch;
			delete p; // 删除
			return q;
		}
		else { // 否则,将左儿子的子孙中最大的节点提上来
			node *q;
			for(q=p->lch; q->rch->rch; q=q->rch); // 往左儿子搜索最大节点
			node *r = q->rch; // r指向左儿子最大子孙节点,q指向最大儿子的父亲
			q->rch = r->lch; // 因为r为提上去的节点,将r的左儿子(有的话,否则为NULL)挂到q的右边
			r->lch = p->lch;
			r->rch = p->rch;
			delete p; // 删除
			return r;
		}
	}
	return p;
}
/**************Usage**************/
	node *testbst=NULL; // 初始化
	testbst = insert(testbst, x); // 插入数值x
	if(find(testbst, x)) // 查找数值x
		// balabala
	else
		// balabala
	testbst = remove(testbst, x); // 删除数值x
\end{lstlisting}

\section{并查集}
并查集是一种用来管理元素分组情况的数据结构,可以高效地进行如下两种操作:
\begin{itemize}
		\item 合并两个集合
		\item 查找某元素属于哪个集合
\end{itemize}

时间复杂度:\(O(\alpha(n))\)
\begin{lstlisting}
int par[MAX_N];
// int height[MAX_N];
void init(int n) { // 初始化
	for (int i = 1; i <= n; ++i) {
		par[i] = i;
		// height[i] = 0;
	}
}
int find(int x) { // 查找根节点(集合)+路径压缩
	return x==par[x]?x:par[x]=find(par[x]);
}
void unite(int x, int y) { // 合并集合
	x = find(x);
	y = find(y);
	if(x!=y) {
	par[x] = y;
	// if(height[x] < height[y])
	// par[x] = y;
	// else
	// par[y] = x;
	// if(height[x] == height[y]) ++height[x];
	}
}
bool same(int x,int y) { // 判断两个元素是否同集合
	return find(x) == find(y);
}
\end{lstlisting}


\chapter{图}
\section{邻接表}
\subsection{样例1}
\begin{lstlisting}
#define MAX_V 100
vector<int> G[MAX_V];

/* 边上有属性
 * struct edge { int to, cost; };
 * vector<edge> G[MAX_V];
 */

int main()
{
	int V, E;
	cin >> V >> E;
	for(int i=0; i<E; ++i) {
		int s, t;
		cin >> s >> t;
		G[s].push_back(t); // s->t
		// G[s].push_back(edge(t, c));
		// G[t].push_back(s); // 无向图
	}
	// balabala...
}
\end{lstlisting}


\subsection{样例2}
\section{单源最短路}
\subsection{Bellman-Ford算法$O(V*E)$}
\begin{lstlisting}
struct Edge{ // 边
	int from, to, cost; // 顶点from指向to权值为cost
} edge[MAX_E];
int d[MAX_V];
int V, E;  // 节点数量V, 边的数量E

bool bellman_ford(int s) { // 求解顶点s出发到所有节点的最短距离
	memset(d, 0x3f, sizeof(d)); // 初始化到INF
	d[s] = 0;
	for(int i=1; i<=V-1; ++i) // 图的顶点编号从1开始计算
		for(int j=1; j<=E; ++j) {
			Edge e=edge[j];
			if(d[e.to] > d[e.from] + e.cost) // 松弛计算
				d[e.to] = d[e.from] + e.cost;
		}

	int flag = true; // 判断有没有负圈
	for(int j=1; j<=E; ++j)
		if(d[edge[j].to] > d[edge[j].from] + edge[j].cost) {
			flag = false;
			break;
		}
	return flag;
}
\end{lstlisting}

\subsection{Dijkstra算法}
$O(V^2)$
\begin{lstlisting}
int cost[MAXV][MAXV]; // cost[u][v]表示e={u,v}的权值(不存在则INF)
int d[MAXV]; // 顶点s出发的最短距离
bool used[MAXV]; // 标记已经使用过的顶点
int V, E; // 顶点数V,边数E

void dijkstra(int s) { // 源点s
	memset(d, 0x3f, sizeof(d)); // 初始化至INF
	memset(used, 0, sizeof(used)); // 初始化至INF
	d[s] = 0;

	while(true) {
		int v = -1;
		for(int u=1; u<=V; ++u) // 从未使用过的节点中选择一个距离最小的顶点,编号从1开始
			if(!used[u] && (v==-1 || d[u] < d[v])) v = u;
		if(v == -1) break; // 已经用完所有顶点了
		used[v] = true; // 标记顶点
		for(int u=1; u<=V; ++u) // 顶点编号从1开始计算
			d[u] = min(d[u], d[v]+cost[v][u]);
	}
}
\end{lstlisting}

% \clearpage
$O(E logV)$
\begin{lstlisting}
struct edge { // 顶点属性
	int to, val;
	edge(int t, int v): to(t), val(v){}
	bool operator<(const edge &b) const {
		return val > b.val;
	}

};
vector<edge> G[MAX_V]; // 邻接链表图
int d[MAX_V];
int V, E; // 顶点数V, 边数E

void dijkstra(int s) {
	priority_queue<edge> que;
	memset(d, 0x3f, sizeof(d));
	d[s] = 0;
	que.push(edge(s, 0)); // 源点入队

	while(!que.empty()) {
		edge p = que.top(); que.pop();
		int v = p.to;
		if(d[v] < p.val) continue; // 当前最小值不是最短距离的话,丢弃
		for(int i=0; i<G[v].size(); ++i) {
			edge e = G[v][i];
			if(d[e.to] > d[v] + e.val) {
				d[e.to] = d[v] + e.val;
				que.push(edge(e.to, d[e.to]));
			}
		}
	}
}
\end{lstlisting}



\section{任意两点间的最短路}
Floyd $O(V^2)$
$$
d[i][j] = min(d[i][j], d[i][k] + d[k][j])
$$

和Bellman-Ford算法一样可以处理负圈,只需检查d[i][i]是否负数的顶点即可。

\begin{lstlisting}
int V, E; // 顶点数V, 边数E
int d[102][102]; // d[u][v]表示边e={u, v}的权值(不存在则设为INF, d[u][u]=0)

void floyd() {
	for(int k=0; k<V; ++k) // 顶点依次从0-V-1开始
		for(int i=0; i<V; ++i)
			for(int j=0; j<V; ++j)
				d[i][j] = min(d[i][j], d[i][k]+d[k][j]); // i到j的最短距离等于i到j的距离与i到k和k到j的距离的最小值
	return;
}
\end{lstlisting}


\end{document}
